\documentclass[12pt]{article}
%%%
\usepackage{proof}
\usepackage{graphicx}
\usepackage{fleqn}
\usepackage{latexsym}
\usepackage{listings}
\usepackage{theorem}
\theoremstyle{break}
\theorembodyfont{\rm}
\setlength{\topmargin}{0cm}
\setlength{\headheight}{0cm}
\setlength{\headsep}{0cm}
\setlength{\oddsidemargin}{0cm}
\setlength{\evensidemargin}{0cm}
\setlength{\textheight}{23cm}
\setlength{\textwidth}{17cm}
%
%%% Definition of newcommands
%
%\newcommand{\boldcal}{\boldmath\mathcal}
%\newcommand{\boldcal}{\boldmath\cal}
%\newcommand{\bottom}{\mbox{$\perp$}}
%\newcommand{\predicateP}{\mbox{\boldmath P }}
%\newcommand{\hatP}{\hat{\mbox{\boldmath P}}}
\newcommand{\procdef}{\stackrel{\rm def}{=}}
%\newcommand{\wff}{{\em wff}}
%\newcommand{\cont}{\hfill{$\Box$}\newline}
%\newcommand{\triagleq}{\buildrel\scriptscriptstyle\triangle\over=}
%\newcommand{\saA}{{\cal A}}
%\newcommand{\transition}[2]{\stackrel{#1}{\rightarrow}_{#2}}
%\newcommand{\trans}[3]{\mbox{$#1\stackrel{#2}{\rightarrow}#3$}}
\newcommand{\trans}[1]{\stackrel{#1}{\rightarrow}}
\newcommand{\transl}[1]{\stackrel{#1}{\longrightarrow}}
\newcommand{\bijection}[1]{\mbox{$\overline{#1}$}}
%\newcommand{\Low}{\mbox{$\cal L$}}
%\newcommand{\High}{\mbox{$\cal H$}}
\newcommand{\thref}[1]{\mbox{Lemma~\ref{#1}}}
\newcommand{\Thref}[1]{\mbox{Lemma~\ref{#1}}}
\newcommand{\figref}[1]{\mbox{Fig.~\ref{#1}}}
\newcommand{\Figref}[1]{\mbox{Fig.~\ref{#1}}}
\newcommand{\secref}[1]{\mbox{Section~\ref{#1}}}
\newcommand{\Secref}[1]{\mbox{Section~\ref{#1}}}
\newcommand{\tblref}[1]{\mbox{Table~\ref{#1}}}
\newcommand{\Tblref}[1]{\mbox{Table~\ref{#1}}}
\newcommand{\eqref}[1]{\mbox{(\ref{#1})}}
\newcommand{\Eqref}[1]{\mbox{(\ref{#1})}}
\newcommand{\emptyseq}{\mbox{$\langle\rangle$}}
\newcommand{\implies}{\ni}
\newcommand{\ltldiammond}{\mbox{$\Diamond$}}
\newcommand{\ltluntil}{\mbox{$\,\,{\cal U}\,$}}
\newcommand{\ltlnext}{\mbox{$\circ\,$}}
\newcommand{\ltlbox}{\mbox{$\Box$}}
\newcommand{\true} {\mbox{${\tt tt}$}}
\newcommand{\false}{\mbox{${\tt ff}$}}
\newcommand{\suffix}[2]{\mbox{\(#1^{#2..}\)}}
\newcommand{\rsquarebracket}{\mbox{$|\hspace{-.8mm}{]}$}}
\newcommand{\lsquarebracket}{\mbox{$[\hspace{-.8mm}{|}$}}
\newcommand{\done}{\hfill{$\blacksquare$}}
\newcommand{\buchi}{\mbox{\rm B\"{u}chi }}
\def\nottrans{\not\Rightarrow}
\def\notRightarrow{\not\Rightarrow}
\def\notin{\not\in}
\def\notmodels{\not\models}
%
\newcounter{nnn}
\newcommand{\itemnum}{\thennn \addtocounter{nnn}{1}}
\setcounter{nnn}{1}
\newcommand{\means}[1]{\mbox{$[\hspace{-.033in}|#1]\hspace{-.045in}|$}}
%----- definition of environments -----
%----- modification of style-files -----
\makeatletter
  \renewcommand{\theequation}{% Modify the number of equations
    \thechapter.\arabic{equation}}
  \@addtoreset{equation}{chapter}
\makeatother

\def\topfraction{1.0}
\def\bottomfraction{1.0}

%-- Copyright --
\makeatletter
\def\copyright{\@ifnextchar[{\@@Copyright}{\@Copyright}}
\def\@@Copyright[#1]#2{\def\slide@copyright{#2}%
                    \def\slidecopyright{#1}}
\def\@Copyright#1{\def\slide@copyright{#1}
               \def\slidecopyright{#1}}
\def\slide@copyright{\hspace*{1pt}}
\def\slidecopyright{\hspace*{1pt}}
\makeatother
%-- one column <- two column --
\makeatletter
\def\@maketitle{\newpage
 \null
 \vskip 2em \begin{center}
 {\LARGE \@title \par} \vskip 1.5em {\large \lineskip .5em
\begin{tabular}[t]{c}\@author
 \end{tabular}\par} 
  {\bf \vspace{1.5em}Abstract\vspace{-.5em}\par\relax} 
  {\vskip 1em \small\usebox{\abstractbox}\par\relax}
 \vskip 1em {\large \@date} \end{center}
 \par
 \vskip 1.5em} 
\makeatother
\newbox\abstractbox
\def\abstract{\global\setbox\abstractbox=\hbox\bgroup
  \begin{minipage}{0.7\textwidth}}
\def\endabstract{\end{minipage}\egroup}

\def\adag{${}^\dag$}% dagger for \author
%%%
%%%
%%%
\newtheorem{definition}{DEFINITION}[section]

%----------------------------------------------------------
\title{NHK$^{\sharp}$ User's Manual}
\author{Noayuki Nagatou\thanks{PRESYSTEMS Inc. Copyright 2014 Naoyuki Nagatou}\\%
\small PRESYSTEMS Inc.\\[6pt]
\small nagatou@presystems.xyz}
\date{}
%----------------------------------------------------------
\markboth{Copyright 2014 Naoyuki Nagatou}{Copyright 2014 Naoyuki Nagatou}

%
%%% ABSTRACTION
%
\begin{abstract}
\end{abstract}

\begin{document}
\maketitle
\section{Installation}
  Compiling those files needs glib-1.2. You need to install it before doing this.
Please refer the manual to install glib-1.2.
  \begin{enumerate}
\item After installing glib-1.2, \\
    \verb|>| tar -xvzf rccs.tar.gz
\item type the following\\
    \verb|>| cd rccs/src
\item  similarly,\\
    \verb|>| make rccs
  \end{enumerate}
If you cannot compile then please edit the make file for your configuration.
Perhaps, glib is installed into a different place.
\section{Syntax of Model Description Language}
\subsection{Primitive Types}
RCCS has two types, integers and strings type.
Operator {\tt +,-,/,*} are defined over these types.
\subsection{Special Actions and Processes}
Action {\tt key} and {\tt display} are special actions. 
{\tt key} is used as an input action, and {\tt display} is used as an output action.
Process {\tt ZERO} and {\tt STOP} are a special process that means do nothing, not terminate the whole.
\subsection{Scope}
In expression (define P (x) body), the scope of x becomes body.
In expression (\bijection{a}(x):body), the scope of x becomes body.
\subsection{Lexical Binding Rule}
Dynamic binding.
\subsection{Syntax of the Discription Language}
%\onecolumn
\footnotesize
  \begin{verbatim}
Agent_Exp     ::= ( define ID ( ID_Seq ) Agent_Exp )
               |  ( define ID ( ) Agent_Exp )
               |  ( bind ID Strings )
               |  ( globalvar IDs Strings )
               |  ( if ( B_exp ) Agent_Exp Agent_Exp )
               |  ( A_Binary_Exp )
               |  (  )
A_Binary_Exp  ::= A_Binary_Exp ++ A_Label_Exp
               |  A_Binary_Exp || A_Label_Exp
               |  A_Label_Exp
A_Label_Exp   ::= A_Label_Exp [ ID_Seq ]
               |  A_Label_Exp { Relabel_Seq }
               |  A_Unary_Exp
A_Unary_Exp   ::= ID ( Value_Seq ) : A_Unary_Exp
               |  ID : A_Unary_Exp
               |  ~ ID ( Value_Seq ) : A_Unary_Exp
               |  ~ ID : A_Unary_Exp
               |  ID ( Value_Seq )
               |  ID
               |  ( Agent_Exp )
ID_Seq        ::= ID_Seq , ID
               |  ID_Seq , ~ ID
               |  ID
               |  ~ ID
Value_Seq     ::= Value_Seq , B_Exp
               |  B_Exp
Relabel_Seq   ::= Relabel_Seq  , ID / ID
               |  Relabel_Seq  , ~ ID / ~ ID
               |  ID / ID
               |  ~ ID / ~ ID
B_Exp         ::= B_Exp | C_Exp
               |  B_Exp & C_Exp
               |  C_Exp
C_Exp         ::= C_Exp < V_Exp
               |  C_Exp <= V_Exp
               |  C_Exp > V_Exp
               |  C_Exp >= V_Exp
               |  C_Exp = V_Exp
               |  V_Exp
V_Exp         ::= V_Exp + V_Term
               |  V_Exp - V_Term
               |  V_Term
V_Term        ::= V_Term * V_Unary_Exp
               |  V_Term / V_Unary_Exp
               |  V_Term % V_Unary_Exp
               |  V_Unary_Exp
V_Unary_Exp   ::= ! V_Unary_Exp
               |  Fact
Fact          ::= Iconst | Strings | TRUE | FALSE | ID
               |  ( B_Exp )
  \end{verbatim}

\subsection{RCCS Operatinal Semantics}\label{OPERATINAL_SEMANTICS}
%\onecolumn
In this section, we define the semantics of the model discription language.
For this purpose, we define an abstract machine.
We define context \(k\).
The following grammar defines a set of contexts.
The context contains a hole, written in \hole, in the place of one subexpression.
  \begin{displaymath}
    \begin{array}{lcl}
k     &::=& \hole \\
      & | & P\|\hole\\
      & | & \hole\|P
    \end{array}
  \end{displaymath}
$k[P]$ means to replace the hole in $k$ with $P$, where P is a process defined in \secref{sec:RCCSsyntax}.
Moreover, we define a function $env$ that maps all free variables to closures.
This function is called a environment, and a closure is a pair of an expression and a environment.
environment \(env\) and colosure \(c\) have mutually recursive definitions.
  \begin{displaymath}
    \begin{array}{rcl}
env &::=& \mbox{a list of pairs }\langle(X,c),\cdots\rangle\\
c &::=& \{(P,env)|FV(P)\subset \dom(env)\}\\
%v &::=& \{(v,env)|FV(P)\in c\}
    \end{array}
  \end{displaymath}
\(env[X\leftarrow c]\) means that \((X,c)\) is added into \(env\), that is, 
\(\{(X,c)\}\cup\{(Y,c')|(Y,c')\in env \mbox{ and } X\neq Y\}\).

In addition to the above definition, we use the following sets.
  \begin{displaymath}
    \begin{array}{rcl}
ch &::=&\mbox{a list of pairs }\langle(Name,\langle m_1,\cdots,m_n\rangle),\cdots\rangle
    \end{array}
  \end{displaymath}
Channel ch represents a set of channels with which processes communicate each other.
\(ch[a\leftarrow v]\) means that \((a,v)\) is added into \(ch\), that is, 
\(\{(a,v)\}\cup\{(b,v')|(b,v')\in ch \mbox{ and } a\neq b\}\).

A state of the machine is a triple \([(exp,env),k,ch]\) which is appended channel \(ch\) to machines in \cite{Felleisen:2002}.
We define single steps of an evaluation function for the description language.
A bijection function on names assocites a name \(a\) to a co-name, written with \(\bijection{a}\).
Notice that \(\bijection{\bijection{\alpha}}=\alpha\).
\(env[\langle x\rangle\leftarrow\langle v\rangle,\langle y\rangle\leftarrow\langle w\rangle,\cdots]\) means \((env[\langle x\rangle\leftarrow \langle v\rangle])[\langle y\rangle\leftarrow\langle w\rangle]\cdots\), and
\(ch[\alpha\leftarrow \langle v\rangle,\beta\leftarrow \langle w\rangle,\cdots]\) means \((ch[\alpha\leftarrow\langle v\rangle])[\beta\leftarrow\langle w\rangle],\cdots\).
We show the operational semantics of RCCS in \figref{fig:semRCCS}
\begin{figure}[tb]
\scriptsize
  \begin{tabular}{ll}
\multicolumn{2}{c}{
\inference[Output(1)]
{(a\leftarrow\langle v\rangle)\not\in ch}
{[(\sim a(v)\colon P,env),\hole,ch]\trans{\sim a}[(P,env),\hole,ch[a\leftarrow\langle v\rangle]]}
}\\\\
\multicolumn{2}{c}{
\inference[Output(2)]
{(a\leftarrow\langle v\rangle)\not\in ch}
{[(\sim a(v)\colon P,env_1),k[(Q,env_2)\|\hole],ch]\trans{\sim a}[(Q,env_2),k[\hole\|(P,env)],ch[a\leftarrow\langle v\rangle]]}
}\\\\
\multicolumn{2}{c}{
\inference[Output(3)]
{(a\leftarrow\langle v\rangle)\not\in ch}
{[(\sim a(v)\colon P,env_1),k[\hole\|(Q,env_2)],ch]\trans{\sim a}[(Q,env_2),k[(P,env_1)\|\hole],ch[a\leftarrow\langle v\rangle]]}
}\\\\
\multicolumn{2}{c}{
\inference[Input]
{(a\leftarrow\langle v\rangle)\in ch}
{[(a(x)\colon P,env),k,ch[a\leftarrow\langle v\rangle]]\trans{a}[k[(P,env[\langle x\rangle\leftarrow\langle v\rangle])],\hole,ch]}
}\\\\
\multicolumn{2}{c}{
\inference[Sum(1)]
{[(P,env_{1}),k,ch]\trans{\alpha}[(P',env_{1}'),k',ch']}
{[(P,env_{1})\mbox{++}(Q,env_2),k,ch]\trans{\alpha}[(P',env_{1}'),k',ch']}
} \\\\
\multicolumn{2}{c}{
\inference[Sum(2)]
{[(Q,env_{2}),k,ch]\trans{\alpha}[(Q',env_{2}'),k',ch']}%\\
% P\not\in\derivative{[(P,env_{1}),k,ch]}}
{[((P,env_{1})\mbox{++}(Q,env_{2}),k,ch]\trans{\alpha}[(Q',env_{2}'),k',ch']}
}\\\\
\multicolumn{2}{c}{
\inference[Com(1)]
{[(P,env_1),k,ch]\trans{\alpha}[(P',env_{1}'),k[\hole\|Q],ch']}
{[(P,env_1)||(Q,env_2),k,ch]\trans{\alpha}[(P',env'),k[\hole\|Q],ch']}
}\\\\
\multicolumn{2}{c}{
\inference[Com(2)]
{[(Q,env_2),k,ch]\trans{\alpha}[(Q',env_{2}'),k[P\|\hole],ch']}%\\
% P\not\in\derivative{[(P,env_{1}),k,ch]}}
{[((P,env_1)||(Q,env_2),k,ch]\trans{\alpha}[(Q',env_{2}'),k[P\|\hole],ch']}
}\\\\
\multicolumn{2}{c}{
\inference[Com(3)]
{[(P,env_{1}),k[\hole\|(Q,env_{2})],ch]\trans{\overline{\alpha}}[(P',env_{1}'),k',ch']\\
 [(Q,env_{2}),k[(P,env_{1})\|\hole],ch]\trans{\alpha}[(Q',env_{2}'),k'',ch'']}
{[(P,env_1)\|(Q,env_2),k,ch]\trans{(\overline{\alpha},\alpha)}[(k[(P',env_{1}')\|(Q',env_{2}')]),\hole,ch]}
}\\\\
\multicolumn{2}{c}{
\inference[Instance]
{[(A(v),env[A\leftarrow P(x)]),k,ch]}
{[(A(v),env[A\leftarrow P(x)]),k,ch]\trans{}[(k[(P,env[A\leftarrow P][\langle x\rangle\leftarrow\langle v\rangle])]),\hole,ch]}
}\\\\
\multicolumn{2}{c}{
\inference[Constant]
{[(A(v),env[A\leftarrow P(x)]),k,ch]\trans{}[(P,env[A\leftarrow P(x)][\langle x\rangle\leftarrow\langle v\rangle]),k,ch]\\
[(P(x),env[A\leftarrow P][\langle x\rangle\leftarrow\langle v\rangle]),k,ch]\trans{\alpha}[(P',env[A\leftarrow P][\langle x\rangle\leftarrow\langle v\rangle]),k,ch]}
{[(A(v),env[A\leftarrow P]),k,ch]\trans{\alpha}[(P',env[A\leftarrow P][\langle x\rangle\leftarrow\langle v\rangle]),k,ch]}
}\\\\
%\multicolumn{2}{c}{
%\inference[Def]
%{}
%{[(\mbox{define}\ A\ P,env),k,ch]\trans{}[(\mbox{ZERO},env[A\leftarrow P]),k,ch]}
%}\\\\
\inference[If(1)]
{\mbox{eval-val}(B,env)\\ [(T,env),k,ch]\trans{\alpha}[(T',env'),k',ch']}
{[({\tt if}\ B\ T\ E,env),k,ch]\trans{\alpha}[(T',env'),k',ch']}
&
\inference[If(2)]
{\neg\mbox{eval-val}(B,env)\\[(E,env),k,ch]\trans{\alpha}[(E',env'),k',ch']}
{[({\tt if}\ B\ T\ E,env),k,ch]\trans{\alpha}[(E',env'),k',ch']}
\\\\
\multicolumn{2}{c}{
\inference[Res]
{\alpha,\bijection{\alpha}\not\in\{\alpha,\cdots\} & [(P,env),k,ch]\trans{\alpha}[(P',env'),k',ch']}
{[(P[\alpha,\cdots],env),k,ch]\trans{\alpha}[(P'[\alpha,\cdots],env'),k',ch']}
}\\\\
\multicolumn{2}{c}{
\inference[Rel(1)]
{[(P,env),k,ch]\trans{\alpha}[(P',env'),k',ch[\alpha\leftarrow\langle v\rangle]]}
{[(P\{\alpha'\slash\alpha,\cdots\},env),k,ch]\trans{\alpha '}[(P'\{\alpha'\slash\alpha,\cdots\},env'),k',ch[\alpha'\leftarrow\langle v\rangle]]}
}\\\\
\multicolumn{2}{c}{
\inference[Rel(2)]
{[(P,env),k,ch[\alpha\leftarrow\langle v\rangle]]\trans{\alpha}[(P',env'),k',ch]}
{[(P\{\alpha'\slash\alpha,\cdots\},env),k,ch[\alpha\leftarrow\langle v\rangle]]\trans{\alpha '}[(P'\{\alpha'\slash\alpha,\cdots\},env'),k',ch]}
}\\\\
%\infer[\mbox{Bind}]
%{[\mbox{bind}\ A\ string,s]\trans{}[\mbox{STOP},s]}
%{}
%\\\\
\inference[ZERO(1)]
{}
{[(\mbox{ZERO},env),\hole,ch]\not\trans{}}
&
\inference[ZERO(2)]
{}
{[(\mbox{ZERO},env),k,ch]\trans{}[k[(\mbox{ZERO},env)],\hole,ch]}
%\\\\
%\inference[Stop(2)]
%{[\mbox{STOP},s]\not\trans{} & [\mbox{STOP},s]\not\trans{}}
%{[\mbox{STOP}||\mbox{STOP},s]\not\trans{}}\\
%\infer[\mbox{Stop}_3]{[\mbox{STOP}++P,s]\trans{\alpha}[P',s']}
%                     {[P,s]\trans{\alpha}[P',s']}
%&
%\infer[\mbox{Stop}_4]{[\mbox{STOP}++\mbox{STOP},s]\not\trans{}}
%                     {[\mbox{STOP},s]\not\trans{}
%                      & [\mbox{STOP},s]\not\trans{}}
  \end{tabular}
\caption{Semantics of RCCS}
\label{fig:semRCCS}
\end{figure}

\section{Semantics of Property Description Language}
\subsection{Special Processes and Port Names}
\section{Coroutine-Like Sequencing}
An important application of coroutine is discrete event simulation, where coroutine may be used to simulate parallel processes within the framework of a sequential program.
\section{Syntax of Formulae}
We use LTL to describe goal properties of processes.
We first assume that a trace has initial states and is a finite sequence of states.
We write the length of trace \(\sigma=s_0\,s_1\,\cdots\,s_n\) to \(|\sigma|\) in which \(|\sigma|\) is \(n+1\).
We write the suffix of \(\sigma=s_0\,s_1\,\cdots\,s_i\,\cdots\,s_n\) starting at \(i\) as \(\suffix{\sigma}{i}=s_i\,\cdots\,s_n\), and the \(i^{\mbox{th}}\) state as \(\sigma^i\).

We assume a vocabulary \(x,y,z,\cdots\) of variables for data values.
For each state, variables are assigned to a single value.
A state formula is any well-formed first-order formula constructed over the given variables.
Such state formulas are evaluated on a single state to a boolean value.
If the evaluation of state formula \(p\) becomes true over \(s\), then we write \(s\lsquarebracket p\rsquarebracket=\true\) and say that \(s\) satisfies \(p\), where $\true$ and $\false$ are truth values, denoting \(true\) and \(false\) respectively.
Let $\varphi$ and $\psi$ be temporal formulas, a temporal formula is inductively constructed as follows:
  \begin{itemize}
\item a state formula is a temporal formula,
\item the negation of a temporal formula \(\neg\varphi\) is a temporal formula,
\item \(\varphi\vee\psi\) and \(\varphi\wedge\psi\) are temporal formulas, and
\item \(\ltlbox\,\varphi\), \(\ltldiammond\,\varphi\), \(\ltlnext\,\varphi\), and \(\varphi\ltluntil\psi\) are temporal formulas.
  \end{itemize}
\section{Semantics of Formulae}
We next define two semantics of temporal formulas over a finite trace according to \cite{Eisner:2003}.
If trace \(\sigma\) satisfies property \(\varphi\), then we write \(\sigma\models\varphi\).
\subsection{Strong Semantics}
Furthermore,
  \begin{itemize}
\item if $p$ is a state formula, then \(\sigma\models p\) iff \(\sigma^0\lsquarebracket p\rsquarebracket=\true\) and \(|\sigma|\neq 0\),
\item \(\sigma\models\neg\varphi\) iff \(\sigma\not\models\varphi\),
\item \(\sigma\models\varphi\vee\psi\) iff \(\sigma\models\varphi\) or \(\sigma\models\psi\),
\item \(\sigma\models\varphi\wedge\psi\) iff \(\sigma\models\varphi\) and \(\sigma\models\psi\),
\item \(\sigma\models\ltlbox\varphi\) iff for all \(0\leq i<|\sigma|\), \(\suffix{\sigma}{i}\models\varphi\),
\item \(\sigma\models\ltldiammond\varphi\) iff there exists \(0\leq i<|\sigma|\) such that \(\suffix{\sigma}{i}\models\varphi\),
\item \(\sigma\models\ltlnext\varphi\) iff \(\sigma'\models\varphi\) where \(\sigma'=\sigma\) if \(|\sigma|=1\) and \(\sigma'=\suffix{\sigma}{1}\) if \(|\sigma|>1\),
\item \(\sigma\models\varphi\ltluntil\psi\) iff there exists \(0\leq k<|\sigma|\) s.t. \(\sigma\models\psi\) and for all \(j<k\), \(\sigma\models\varphi\).
  \end{itemize}

A formula \(\varphi\) is satisfiable if there exists a sequence \(\sigma\) such that \(\sigma\models \varphi\).
Given set of traces $T$ and formula \(\varphi\), \(\varphi\) is valid over $T$ if for all \(\sigma\in T\), \(\sigma\models\varphi\).
\subsection{Weak Semantics}
Furthermore,
  \begin{itemize}
\item if $p$ is a state formula, then \(\sigma\models p\) iff \(\sigma^0\lsquarebracket p\rsquarebracket=\true\) or \(|\sigma|=0\),
\item \(\sigma\models\neg\varphi\) iff \(\sigma\not\models\varphi\),
\item \(\sigma\models\varphi\vee\psi\) iff \(\sigma\models\varphi\) or \(\sigma\models\psi\),
\item \(\sigma\models\varphi\wedge\psi\) iff \(\sigma\models\varphi\) and \(\sigma\models\psi\),
\item \(\sigma\models\ltlbox\varphi\) iff for all \(0\leq i<|\sigma|\), \(\suffix{\sigma}{i}\models\varphi\),
\item \(\sigma\models\ltldiammond\varphi\) iff there exists \(0\leq i<|\sigma|\) such that \(\suffix{\sigma}{i}\models\varphi\),
\item \(\sigma\models\ltlnext\varphi\) iff \(\sigma'\models\varphi\) where \(\sigma'=\sigma\) if \(|\sigma|=1\) and \(\sigma'=\suffix{\sigma}{1}\) if \(|\sigma|>1\),
\item \(\sigma\models\varphi\ltluntil\psi\) iff there exists \(0\leq k<|\sigma|\) s.t. \(\sigma\models\psi\) and for all \(j<k\), \(\sigma\models\varphi\).
  \end{itemize}
%
%%% RELATIONSHIPS BETWEEN MODELS AND FORMLAE
%
\section{Relationships between Models and Formlae}
In this subsection, we describe the relationship between algebraic models and LTL formulas.
The modeling language enables us to pass values via input prefix \(\alpha(e)\) and output prefix \(\bijection{\alpha}(x)\) with the same name.
Execution of \(\alpha(e)\) produces  value \(v\) of \(e\).
Execution of \(\bijection{\alpha}(x)\) causes a single assignment to \(x\).
Furthermore, the execution of two actions causes atomic assignment \(x := v\), that is, communication between two agents produces a new state by changing the values of the variables.
This is similar to the first paragraph in Section~3.3 of \cite[page 290]{lamport84hoare}.

This atomic assignment changes states, and  we represent the change as \(s[v/x]\), which denotes a change in the values of \(x\) in \(s\) to \(v\).
A state is a mapping from variables to values.
Assuming that \(\tt Var_E\) is a set of variables that appears in prefixes in agent \(E\) with range \(\tt V\), \(s:Var_E\rightarrow {\tt V}\).
For example, the evaluation \(s\lsquarebracket x=y\rsquarebracket\) of \(x=y\) at \(s\) becomes \(s\lsquarebracket x\rsquarebracket=s\lsquarebracket y\rsquarebracket\), and at \(s[v/x]\), \(s[v/x]\lsquarebracket x\rsquarebracket=s[v/x]\lsquarebracket y\rsquarebracket\), i.e., \(v=s[y]\).

Therefore, communication between agents produces a sequence of assignments, which then produces a sequence of state changes called a trace.
Let a set of traces produced by agent \(E\) be \(T\).
If for all traces \(\sigma\in T\), \(\sigma\models\varphi\), then we state that \(\varphi\) is valid over \(E\) and write \(E\models\varphi\).
%
%%% TRANSITION OF AUTOMATA
%
\subsection{Transition of Automata}
A \buchi automaton $m$ contains of five components:
  \begin{itemize}
\item A finite set of states, denoted $Q$.
\item A finite set of input symboles, denoted $\Sigma$.
\item A transition function $\delta$ that takes a state and an input symbol, and returns a next state.
If $q$ is a state, and $s$ is an input symbol, then \(\delta(q,a)\) returns state $p$.
\item A start state $q_0$ is a state in $Q$.
\item A set of accepting states $Q_{\infty}$ is a subset of $Q$.
  \end{itemize}
In this paper, an input symbol becomes a state of a model.
We talk about an automaton $m$ in \emph{five-tuple} notation: \((Q,\Sigma,\delta,q_0,Q_{\infty})\).

Now, we need to make the notion of the language that an automaton accepts.
To do this, we define an extended transition function.
The extended transition function constructed from $\delta$ is called \(\hat{\delta}\).
We define \(\hat{\delta}\) by induction on the length of an input string $\sigma$, as follows:
  \begin{displaymath}
\hat{\delta}(q,\sigma)=\left\{
      \begin{array}{ll}
\eta & \mbox{if \(|\sigma |=0\)}\\
\delta(\hat{\delta}(q,\sigma^{..n-1}),\sigma^{n}) & \mbox{if \(0<|\sigma |<\omega\)}.
      \end{array}
\right.
  \end{displaymath}
We define the laguage \({\cal L} (m)\) of automaton $m$.
Let \(INF(\sigma)\) be a set of automaton states that appear infinitely often in while reading $\sigma$,
then $\sigma$ is accepted by $m$ if and only if \(INF(\sigma)\cup Q_{\infty}\neq\emptyset\).
Thus,
  \begin{quotation}
    \begin{math}
{\cal L}(m)=\{\rho\mid \rho^0=q^0, \rho^{|\sigma|}=\hat{\delta}(\rho,\sigma)\mbox{, and }INF(\sigma)\cup Q_{\infty}\neq\emptyset\}.
    \end{math}
  \end{quotation}

\(\eta\) depends on weak or strong semantics.
In weak semantics, \(\eta\) is \(q\) that is regarded as an element of \(Q_{\infty}\).
In strong semantics, \(\eta\) is \(\Lambda\), where \(\Lambda\) is inconsistency.
%
%%% CORRESPONDENCE
%
\subsection{Correspondence between Models and Formulae}
We describe a correspondence between a model and a \buchi automaton of a formulae of a property which the model are required.
The correspondece is expressed with Hoare triple: \{P\}$\alpha$\{P'\},
where $P$ and $P'$ are boolean predicates, and $\alpha$ is an action which a model performs.

An automaton $m$ enters an automaton state \(q_j\) if there exists a history containing program state $s$ and $m$ is transformed from \(q_i\) into \(q_j\) by reading $s$.
We define \emph{correspondence invariant} by induction \cite{alpern:1987}.
\begin{definition}[Correspondece Basis]
  \begin{quotation}
\(\forall i: q_j\in Q\mbox{ and }(Init_{\pi}\wedge T_{0j})\Rightarrow C_j\),
  \end{quotation}
where \(Init_{\pi}\) is the initial states of model \(\pi\).
\end{definition}
\begin{definition}[Correspondece Induction]
  \begin{quotation}
\(\forall \alpha:\forall i: \alpha\in A\cup\bijection{A}\mbox{ and }q_i\in Q\mbox{ and }
\{C_i\}\alpha\{\wedge_{q_j\in Q}(T_{ij}\Rightarrow C_j)\}\).
  \end{quotation}
\end{definition}
%
%%% BIBLIOGRAPHY
%
\bibliographystyle{alpha}
\bibliography{reference}

\end{document}

