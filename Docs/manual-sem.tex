%\onecolumn
In this section, we define the semantics of the model discription language.
For this purpose, we define an abstract machine.
We define context \(k\).
The following grammar defines a set of contexts.
The context contains a hole, written in \hole, in the place of one subexpression.
  \begin{displaymath}
    \begin{array}{lcl}
k     &::=& \hole \\
      & | & P\|\hole\\
      & | & \hole\|P
    \end{array}
  \end{displaymath}
$k[P]$ means to replace the hole in $k$ with $P$, where P is a process defined in \secref{sec:RCCSsyntax}.
Moreover, we define a function $env$ that maps all free variables to closures.
This function is called a environment, and a closure is a pair of an expression and a environment.
environment \(env\) and colosure \(c\) have mutually recursive definitions.
  \begin{displaymath}
    \begin{array}{rcl}
env &::=& \mbox{a list of pairs }\langle(X,c),\cdots\rangle\\
c &::=& \{(P,env)|FV(P)\subset \dom(env)\}\\
%v &::=& \{(v,env)|FV(P)\in c\}
    \end{array}
  \end{displaymath}
\(env[X\leftarrow c]\) means that \((X,c)\) is added into \(env\), that is, 
\(\{(X,c)\}\cup\{(Y,c')|(Y,c')\in env \mbox{ and } X\neq Y\}\).

In addition to the above definition, we use the following sets.
  \begin{displaymath}
    \begin{array}{rcl}
ch &::=&\mbox{a list of pairs }\langle(Name,\langle m_1,\cdots,m_n\rangle),\cdots\rangle
    \end{array}
  \end{displaymath}
Channel ch represents a set of channels with which processes communicate each other.
\(ch[a\leftarrow v]\) means that \((a,v)\) is added into \(ch\), that is, 
\(\{(a,v)\}\cup\{(b,v')|(b,v')\in ch \mbox{ and } a\neq b\}\).

A state of the machine is a triple \([(exp,env),k,ch]\) which is appended channel \(ch\) to machines in \cite{Felleisen:2002}.
We define single steps of an evaluation function for the description language.
A bijection function on names assocites a name \(a\) to a co-name, written with \(\bijection{a}\).
Notice that \(\bijection{\bijection{\alpha}}=\alpha\).
\(env[\langle x\rangle\leftarrow\langle v\rangle,\langle y\rangle\leftarrow\langle w\rangle,\cdots]\) means \((env[\langle x\rangle\leftarrow \langle v\rangle])[\langle y\rangle\leftarrow\langle w\rangle]\cdots\), and
\(ch[\alpha\leftarrow \langle v\rangle,\beta\leftarrow \langle w\rangle,\cdots]\) means \((ch[\alpha\leftarrow\langle v\rangle])[\beta\leftarrow\langle w\rangle],\cdots\).
We show the operational semantics of RCCS in \figref{fig:semRCCS}
\begin{figure}[tb]
\scriptsize
  \begin{tabular}{ll}
\multicolumn{2}{c}{
\inference[Output(1)]
{(a\leftarrow\vec{v})\not\in ch}
{[(\sim a(\vec{v})\colon P),env,\hole,ch]\trans{\sim a}[P,env,\hole,ch[a\leftarrow\vec{v}]]}
}\\\\
\multicolumn{2}{c}{
\inference[Output(2)]
{(a\leftarrow\vec{v})\not\in ch}
{[(\sim a(\vec{v})\colon P),env,k[Q\|\hole],ch]\trans{\sim a}[Q,env,k[\hole\|(P,env)],ch[a\leftarrow\vec{v}]]}
}\\\\
\multicolumn{2}{c}{
\inference[Output(3)]
{(a\leftarrow\vec{v})\not\in ch}
{[(\sim a(\vec{v})\colon P),env,k[\hole\|Q],ch]\trans{\sim a}[Q,env,k[(P,env)\|\hole],ch[a\leftarrow\vec{v}]]}
}\\\\
\multicolumn{2}{c}{
\inference[Input]
{}
{[(a(\vec{x})\colon P),env,k,ch[a\leftarrow\vec{v}]]\trans{a}[k[(P,env[\vec{x}\leftarrow\vec{v}])],env,\hole,ch]}
}\\\\
\multicolumn{2}{c}{
\inference[Sum(1)]
{[P,env,k,ch]\trans{\alpha}[P',env',k',ch']}
{[P\mbox{++}Q,env,k,ch]\trans{\alpha}[P',env',k',ch']}
} \\\\
\multicolumn{2}{c}{
\inference[Sum(2)]
{[Q,env,k,ch]\trans{\alpha}[Q',env',k',ch']&P=0}
{[P\mbox{++}Q,env,k,ch]\trans{\alpha}[Q',env',k',ch']}
}\\\\
\multicolumn{2}{c}{
\inference[Com(1)]
{[P,env,k,ch]\trans{\alpha}[P',env',k[\hole\|(Q,env)],ch']}
{[P\|Q,env,k,ch]\trans{\alpha}[P',env',k[\hole\|(Q,env)],ch']}
}\\\\
\multicolumn{2}{c}{
\inference[Com(2)]
{[Q,env,k,ch]\trans{\alpha}[Q',env',k[(P,env)\|\hole],ch']&P=0}
{[P\|Q,env,k,ch]\trans{\alpha}[Q',env',k[(P,env)\|\hole],ch']}
}\\\\
\multicolumn{2}{c}{
\inference[Com(3)]
{[P,env,k[\hole\|(Q,env)],ch]\trans{\overline{\alpha}}[P',env',k',ch']\\
 [Q,env,k[(P,env)\|\hole],ch]\trans{\alpha}[Q',env'',k'',ch'']}
{[P\|Q,env,k,ch]\trans{(\overline{\alpha},\alpha)}[k[P'\|Q'],env''',\hole,ch''']}
}\\\\
\multicolumn{2}{c}{
\inference[Closure]
{[P,env',k,ch]\trans{\alpha}[P',env',k,ch]}
{[(P,env'),env,k,ch]\trans{\alpha}[(P',env'),env,k,ch]}
}\\\\
\multicolumn{2}{c}{
\inference[Constant]
{[A(\vec{v}),env[A\leftarrow P(\vec{x})],k,ch]\trans{}[(P,env[A\leftarrow P(\vec{x})][\vec{x}\leftarrow\vec{v}]),env[A\leftarrow P(\vec{x})],k,ch]\\
[(P(\vec{x}),env[A\leftarrow P][\vec{x}\leftarrow\vec{v}]),env[A\leftarrow P],k,ch]\trans{\alpha}[(P',env[A\leftarrow P(\vec{x})][\vec{x}\leftarrow\vec{v}]),env[A\leftarrow P],k,ch]}
{[A(\vec{v}),env[A\leftarrow P(\vec{x})],k,ch]\trans{\alpha}[(P',env[A\leftarrow P(\vec{x})][\vec{x}\leftarrow\vec{v}]),env[A\leftarrow P(\vec{x})],k,ch]}
}\\\\
\inference[If(1)]
{\mbox{eval-val}(B,env)\\ [T,env,k,ch]\trans{\alpha}[T',env',k',ch']}
{[({\tt if}\ B\ T\ E),env,k,ch]\trans{\alpha}[T',env',k',ch']}
&
\inference[If(2)]
{\neg\mbox{eval-val}(B,env)\\[E,env,k,ch]\trans{\alpha}[E',env',k',ch']}
{[({\tt if}\ B\ T\ E),env,k,ch]\trans{\alpha}[E',env',k',ch']}
\\\\
\multicolumn{2}{c}{
\inference[Res]
{\alpha,\bijection{\alpha}\not\in\{\alpha,\cdots\} & [(P,env),k,ch]\trans{\alpha}[(P',env'),k',ch']}
{[(P[\alpha,\cdots],env),k,ch]\trans{\alpha}[(P'[\alpha,\cdots],env'),k',ch']}
}\\\\
\multicolumn{2}{c}{
\inference[Rel(1)]
{[(P,env),k,ch]\trans{\alpha}[(P',env'),k',ch[\alpha\leftarrow\langle v\rangle]]}
{[(P\{\alpha'\slash\alpha,\cdots\},env),k,ch]\trans{\alpha '}[(P'\{\alpha'\slash\alpha,\cdots\},env'),k',ch[\alpha'\leftarrow\langle v\rangle]]}
}\\\\
\multicolumn{2}{c}{
\inference[Rel(2)]
{[(P,env),k,ch[\alpha\leftarrow\langle v\rangle]]\trans{\alpha}[(P',env'),k',ch]}
{[(P\{\alpha'\slash\alpha,\cdots\},env),k,ch[\alpha\leftarrow\langle v\rangle]]\trans{\alpha '}[(P'\{\alpha'\slash\alpha,\cdots\},env'),k',ch]}
}\\\\
%\infer[\mbox{Bind}]
%{[\mbox{bind}\ A\ string,s]\trans{}[\mbox{STOP},s]}
%{}
%\\\\
\inference[ZERO(1)]
{}
{[(\mbox{ZERO},env),\hole,ch]\not\trans{}}
&
\inference[ZERO(2)]
{}
{[(\mbox{ZERO},env),k,ch]\trans{}[k[(\mbox{ZERO},env)],\hole,ch]}
%\\\\
%\inference[Stop(2)]
%{[\mbox{STOP},s]\not\trans{} & [\mbox{STOP},s]\not\trans{}}
%{[\mbox{STOP}||\mbox{STOP},s]\not\trans{}}\\
%\infer[\mbox{Stop}_3]{[\mbox{STOP}++P,s]\trans{\alpha}[P',s']}
%                     {[P,s]\trans{\alpha}[P',s']}
%&
%\infer[\mbox{Stop}_4]{[\mbox{STOP}++\mbox{STOP},s]\not\trans{}}
%                     {[\mbox{STOP},s]\not\trans{}
%                      & [\mbox{STOP},s]\not\trans{}}
  \end{tabular}
\caption{Semantics of RCCS}
\label{fig:semRCCS}
\end{figure}
